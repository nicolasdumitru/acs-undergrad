\documentclass[a4paper]{article}
\usepackage[a4paper, left=25mm, right=25mm, top=20mm, bottom=20mm]{geometry} % Margins
\usepackage[utf8]{inputenc}

\usepackage{amsmath}
\usepackage{amssymb}
\usepackage{amsthm}
\usepackage[romanian]{babel}
\usepackage{hyperref}

\theoremstyle{definition}
\newtheorem{definition}{Definiție}[section]

\theoremstyle{plain} % default
\newtheorem{theorem}{Teoremă}[section]

\theoremstyle{remark}
\newtheorem*{remark}{Atenție}
\newtheorem*{note}{Notă}
\newtheorem{case}{Caz particular}

\newcommand{\seqlim}{\lim_{n \to \infty}}

\renewcommand{\epsilon}{\varepsilon} % rounder epsilon (style preference)

\title{Analiză Matematică: Rezolvări Lista 1}
\author{Nicolas Dumitru}
\date{\today}

\begin{document}

\maketitle

\textbf{Exercițiul 7.}

a) \(a_n = n - \sqrt{n^2 - n}\);
\begin{gather*}
	\seqlim a_n = \seqlim n - \sqrt{n^2 - n}
	= \seqlim \frac{n^2 - (n^2 - n)}{n + \sqrt{n^2 - n}} \\
	= \seqlim \frac{n}{n\left(1 + \sqrt{1 - \frac{1}{n}}\right)}
	= \seqlim \frac{1}{1 + \sqrt{1 - \frac{1}{n}}} = \frac{1}{2}
\end{gather*}

b) \(a_n = \sqrt{2^n + 3^n}\)
\begin{gather*}
	\seqlim a_n = \seqlim \sqrt[n]{2^n + 3^n}
	= \seqlim \sqrt[n]{3^n \left(\left(\frac{2}{3}\right)^n + 1\right)}
	= 3 \seqlim \sqrt[n]{\left(\left(\frac{2}{3}\right)^n + 1\right)}
	= 3 \cdot 1 = 3
\end{gather*}

c) \(a_n = \left(1 - \frac{7}{n}\right)^n\)
\begin{gather*}
	a_n = \left(1 - \frac{7}{n}\right)^n = \left(\left(1 - \frac{7}{n}\right)^\frac{n}{7}\right)^7 \xrightarrow{n \to \infty} e^7
\end{gather*}


d) \(a_n = \frac{(\ln n)^5}{\sqrt n}\)

Fie funcția \(f: \mathbb R \to \mathbb R, f(x) = \frac{(\ln x)^5}{\sqrt x}\).
\begin{gather*}
	\seqlim a_n = \seqlim \frac{(\ln n)^5}{\sqrt n} = \lim_{x \to \infty} f(x) \overset{\text{l'Hopital (de 5 ori)}}{=} 5! \cdot 2^5 \lim_{x \to \infty} \frac{1}{\sqrt x} = 0
\end{gather*}

\textbf{Exercițiul 8.}
Pentru simplitate, folosindu-ne de modul, luăm \(m = n + p, p \ge 0\) fără pierderea generalității.
\begin{gather*}
	| x_{n + p} - x_n | = \frac{1}{n + 1} + \frac{1}{n + 2} + ... + \frac{1}{n
		+ p} \le \frac{p}{(n + 1)^2} \\
	\frac{p}{(n + 1)^2} < \epsilon \iff n > \sqrt{\frac{\epsilon}{p}} - 1 \implies \\
	\implies \forall \epsilon > 0, \exists N(\epsilon) \text{ a. î. } \forall m
	\ge N(\epsilon), \forall n \ge N(\epsilon), |x_m - x_n| < \epsilon \iff \\
	\iff x_n \text{ e șir Cauchy, deci convergent} \qed
\end{gather*}

\begin{flushright}
	\copyright 2024 Nicolas Dumitru.
	Acest document este licențiat sub licența \href{https://creativecommons.org/licenses/by-sa/4.0/}{Creative Commons Attribution-ShareAlike 4.0 International}.

\end{flushright}

\end{document}
