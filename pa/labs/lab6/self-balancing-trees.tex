\documentclass{article}

\usepackage{enumitem}

\usepackage{amsmath}
\usepackage{amsthm}

\theoremstyle{definition}
\newtheorem{defn}{Definiția}

\title{Arbori echilibrați}
\author{Nicolas Dumitru}

\begin{document}

\maketitle

Un arbore B (numit in continuare B-tree) este arbore echilibrat de cautare
proiectat să funcționeze bine pe medii de stocare secundare. Un B-tree e similar
cu un red-black tree, dar e mai bun pentru minimizarea operatiilor care
acceseaza mediile de stocare. Multe sisteme de baze de date folosesc B-trees sau
variante ale lor pentru a stoca informatii.
B-trees difera de red-black trees prin faptul ca nodurile unui B-tree pot avea
multi copii (deci arborele nu este neaparat binar). Similar cu arborii
rosu-negru, B-trees pot fi folositi pentru operatii pe multimi in timp O(log n)
(unde n e numarul de noduri): cautare, insertie, stergere.

\begin{defn}[B-tree (conform CLRS]
	Un arbore B (B-tree) T este un arbore cu
	rădăcina T.root care are următoarele proprietăți:
	\begin{enumerate}[label=\arabic*.]
		\item Fiecare nod x are următoarele atribute:
		      \begin{enumerate}[label=\alph*.]
			      \item $x.n$, numărul de chei (keys) stocate în nodul $x$,
			      \item cele $x.n$ chei, $x.key_1, x.key_2, \dots, x.key_{x.n}$, stocate în ordine
			            monoton crescătoare, astfel încît
			            $x.key_1 \leq x.key_2 \leq \dots \leq x.key_{x.n}$,
			      \item $x.\mathit{leaf}$, o valoare booleană care e $\mathtt{TRUE}$ dacă $x$ e o frunză și $\mathtt{FALSE}$ dacă $x$
			            e un nod intern.
		      \end{enumerate}

		\item Fiecare nod intern $x$ conține $x.n + 1$ pointeri $x.c_1, x.c_2, \dots, x.c_{x.n + 1}$ către copiii săi. Nodurile frunză nu au copii, deci
		      atributele lor $c_i$ sînt nedefinite.

		\item Cheile x.key separă intervalele de chei stocate în fiecare
		      subtree: dacă $k_i$ e orice cheie în subarborele cu rădăcina $x.c_i$,
		      atunci
		      \begin{equation*}
			      k_1 \leq x.key_1 \leq k_2 \leq x.key_2 \leq \dots \leq x.key_{x.n} \leq k_{x.n + 1} \text{.}
		      \end{equation*}

		\item Toate frunzele au aceeași adîncime, care este înălțimea $h$ a arborelui.

		\item Nodurile au limite inferioare și superioare pentru numărul de chei
		      pe care le pot conține, exprimate în funcție de
		      un întreg fixat $t \geq 2$ numit gradul minim
		      (minimum degree) al arborelui B:
		      \begin{enumerate}[label=\alph*.]
			      \item Fiecare nod cu excepția rădăcinii
			            trebuie să aibă cel puțin $t - 1$ chei.
			            Fiecare nod intern cu excepția rădăcinii are
			            așadar cel puțin $t$ copii. Dacă arborele nu
			            e gol, rădăcina trebuie să aibă cel puțin o
			            cheie.
			      \item Fiecare nod poate conține cel mult
			            $2t- 1$ chei. Prin urmare, un nod intern
			            poate avea cel mult $2t$ copii. Spunem că un
			            nod e \emph{plin} dacă conține exact $2t -
				            1$ chei.
		      \end{enumerate}
	\end{enumerate}
\end{defn}

\begin{defn}[AVL Tree]
	Un \emph{AVL-tree} este un binary search tree echilibrat, care respectă
	următoarele proprietăți:

	\begin{enumerate}
		\item \textbf{Proprietatea de căutare:} Pentru orice nod $v$,
		      \begin{equation*}
			      \forall\, u \in \text{subarborele stîng al
				      lui } v:\; \mathrm{key}(u) <
			      \mathrm{key}(v), \quad \forall\, w \in
			      \text{subarborele drept al lui } v:\;
			      \mathrm{key}(w) > \mathrm{key}(v) \text{.}
		      \end{equation*}
		\item \textbf{Proprietatea de echilibru:} Pentru orice nod $v$, factorul de echilibru
		      \begin{equation*}
			      \mathrm{bf}(v) = \mathrm{height}(v.l) - \mathrm{height}(v.r)
		      \end{equation*}
		      satisface
		      $\mathrm{bf}(v) \in \{-1,0,1\}$, $v.l$ și $v.r$ fiind subarborele stîng și subarborele drept.
		\item \textbf{Menținerea echilibrului:} După fiecare inserție sau ștergere, dacă pentru un nod $v$
		      $\lvert \mathrm{bf}(v)\rvert > 1$, atunci structura arborelui este corectată prin una din cele patru rotații:
		      \begin{itemize}
			      \item Rotire simplă la dreapta (\textit{Right Rotation})
			      \item Rotire simplă la stânga (\textit{Left Rotation})
			      \item Dublă rotire stânga–dreapta (\textit{Left–Right Rotation})
			      \item Dublă rotire dreapta–stânga (\textit{Right–Left Rotation})
		      \end{itemize}
	\end{enumerate}

\end{defn}

Operațiile de căutare, inserare și ștergere într-un B-tree sau într-un AVL-tree
au complexitate $O(\log n)$, unde $n$ este numărul de noduri. Pentru a asigura
această complexitate, orice operație de modificare trebuie să reechilibreze
arborele.

\end{document}
