\documentclass{article}

\usepackage[romanian]{babel}
\usepackage{amsmath}
\usepackage{amsthm}
\usepackage{indentfirst}
\usepackage{graphicx}

\renewcommand{\refname}{Bibliography}

\title{Complexitatea algoritmilor și caracterizarea asimptotică a timpului de rulare}
\author{Nicolas Dumitru}
% \date{\today} % redundant

\begin{document}

\maketitle

\tableofcontents

% introducere - context, structura lucrarii (> 1/2 pg)
\section{Introducere}
Analizarea resursele necesare pentru rularea unui algoritm este critică pentru
determinarea eficienței sale și astfel are o importanță deosebită pentru
descrierea funcționării sale, împreună cu demonstrarea corectitudinii
răspunsurilor date de acestea. Resursele care ne interesează sunt \emph{timpul}
și \emph{memoria}.
pentru analiza eficienței \emph{asimptotice} a algoritmilor,

% literature review/state of the art - studi in domeniu (> 1 pg)
\section{Literature review} % o sa gasim alt titlu (Românesc) pentru sectiunea asta

% aplicatii in lumea reala/use cases (> 1 pg)
\section{Importanța complexității algoritmilor}
Vedem ce are de zis Aaronson pe tema asta (n-am curaj sa-l contrazic).

% concluzii (> 1/2 pg)
\section{Concluzii}

\begin{thebibliography}{ }
	\bibitem{clrs}
	Cormen, Thomas H., et al. Introduction to algorithms. MIT press, 2022.
	\bibitem{qcsd}
	Aaronson, Scott. Quantum computing since Democritus. Cambridge University Press, 2013.
	\bibitem{karp}
	Karp, Richard M. "Reducibility among combinatorial problems." 50 Years of Integer Programming 1958-2008: from the Early Years to the State-of-the-Art. Berlin, Heidelberg: Springer Berlin Heidelberg, 2009. 219-241.
\end{thebibliography}

\end{document}
