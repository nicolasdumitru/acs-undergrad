\documentclass{article}

\usepackage[romanian]{babel}
\usepackage{amsmath}
\usepackage{amsthm}
\usepackage{amssymb}
\usepackage{indentfirst}
\usepackage{graphicx}
\usepackage{enumerate}

\renewcommand{\refname}{Bibliography}

\title{Complexitatea algoritmilor și caracterizarea asimptotică a timpului de rulare}
\author{Nicolas Dumitru}
% \date{\today} % redundant

\begin{document}

\maketitle

\tableofcontents

% introducere - context, structura lucrarii (> 1/2 pg)
\section{Introducere}
Se poate spune că teoria complexității computaționale este între cele mai
însemnate realizări intelectuale ale umanității, împreună cu focul, roata și
teoria computabilității \cite{qcsd}.

Analizarea resursele necesare pentru rularea unui algoritm este critică pentru
determinarea eficienței sale și astfel are o importanță deosebită pentru
descrierea funcționării sale, împreună cu demonstrarea corectitudinii
răspunsurilor date de acestea. Resursele care ne interesează cel mai adesea sunt \emph{timpul}
și \emph{memoria}.


Ordinul de creștere al timpului de rulare al unui algoritm este o metodă simplă
de caracterizare a eficienței sale și ne permite să îl comparăm cu algoritmi
alternativi \cite{clrs}.

% literature review/state of the art - studi in domeniu (> 1 pg)
\section{Literature review} % o sa gasim alt titlu (Românesc) pentru sectiunea asta

\subsection{Notații asimptotice}

\begin{description}
	\item[Notația $O$] \emph{limită superioară} pentru comportamentul asimptotic
	      al unei funcții.
	\item[Notația $\Omega$] \emph{limită inferioară} pentru comportamentul
	      asimptotic al unei funcții.
	\item[Notația $\Theta$] \emph{limită strânsă} pentru comportamentul
	      asimptotic al unei funcții.
\end{description}

Definițiile formale pentru cele trei notații asimptotice sunt:
\begin{enumerate}
	\item Notația $O$
	      \begin{multline}
		      O(g(n)) = \{ f(n) \mid \exists c \in \mathbb{R},
		      \exists n_0 \in \mathbb{N},\; c > 0,\, n > 0 \text{ astfel încât } \\
		      \forall n \geq n_0, 0 \le f(n) \leq c g(n) \}
	      \end{multline}

	\item Notația $\Omega$
	      \begin{multline}
		      \Omega(g(n)) = \{ f(n) \mid \exists c \in \mathbb{R},
		      \exists n_0 \in \mathbb{N},\; c > 0,\, n > 0 \text{ astfel încât } \\
		      \forall n \geq n_0, 0 \le c g(n) \leq f(n) \}
	      \end{multline}

	\item Notația $\Theta$
	      \begin{multline}
		      \Theta(g(n)) = \{ f(n) \mid \exists c_1, c_2 \in \mathbb{R},
		      \exists n_0 \in \mathbb{N},\; c > 0,\, n > 0 \text{ astfel încât } \\
		      \forall n \geq n_0, 0 \le c_1 g(n) \leq f(n) \leq c_2 g(n) \}
	      \end{multline}

	\item Notația $o$
	      \begin{multline}
		      o(g(n)) = \{ f(n) \mid \exists c \in \mathbb{R},
		      \exists n_0 \in \mathbb{N},\; c > 0,\, n > 0 \text{ astfel încât } \\
		      \forall n \geq n_0, 0 \leq f(n) \leq c g(n) \}
	      \end{multline}

	\item Notația $\omega$
	      \begin{multline}
		      \omega(g(n)) = \{ f(n) \mid \forall c > 0,
		      \exists n_0 > 0 \text{ astfel încât } \\
		      0 \leq c g(n) < f(n), \forall n \geq n_0 \}
	      \end{multline}

\end{enumerate}

\subsection{Notații asimptotice în ecuații și inecuații}
Numărul de funcții anonime dintr-o expresie este înțeles ca fiind egal cu
numărul de apariții ale notației asimptotice. De exemplu, în expresia
\begin{equation*}
	\sum_{i = 1}^{n} O(i) \text{,}
\end{equation*}
avem o singură funcție anonimă (functie de $i$). Expresia \emph{nu} este,
deci echivalentă cu $O(1) + O(2) + \cdots + O(n)$, care nu are o interpretare
clară.

% aplicatii in lumea reala/use cases (> 1 pg)
\section{Importanța complexității algoritmilor}
Vedem ce are de zis Aaronson pe tema asta (n-am curaj sa-l contrazic).

% concluzii (> 1/2 pg)
\section{Concluzii}

\begin{thebibliography}{ }
	\bibitem{clrs}
	Cormen, Thomas H., et al. Introduction to algorithms. MIT press, 2022.
	\bibitem{qcsd}
	Aaronson, Scott. Quantum computing since Democritus. Cambridge University Press, 2013.
	\bibitem{karp}
	Karp, Richard M. "Reducibility among combinatorial problems." 50 Years of Integer Programming 1958-2008: from the Early Years to the State-of-the-Art. Berlin, Heidelberg: Springer Berlin Heidelberg, 2009. 219-241.
\end{thebibliography}

\end{document}
